\section{PROGRAM AND DESIGN OF THE RESEARCH INVESTIGATION}
\label{sec:research-design}

\subsection{Objectives, Methodology and Research Plan}

\subsubsection{Objectives}
The main objective of this research is to advance a more accurate and intepretable medical AI model through multimodal learning. Eye tracking data, clinical data and Chest X-ray image are the 3 modalities that will be included in this research.\\*

\noindent
\textbf{AIM 1 (RQ 1 \& RQ 5)}:\\*
Gain better understanding of radiologists classification pattern of CXR images through eye tracking data.And how radiologists perceive other eye tracking classification pattern from other radiologists. \\* 

\noindent
\textbf{AIM 2 (RQ 2 \& RQ 3 \& 4)}:\\*
 Create an explainable human centered multimodal deep learning architecture that supports prediction of lung diseases based on human classification patterns from eye tracking data and pupil dilations, and patients' clinical data.\\*


\noindent
\textbf{AIM 3 (RQ 1 \& RQ 5 \& RQ 6)}:\\*
Design human-centred interactive explanations and interfaces based on probabilistic graphical models that allow medical practitioners to interact, ask questions, and learn the lung disease diagnosis through the explanations \\*

\subsubsection{Methodology}
This research will follow the guidance from Desing Science Research Methodology (DSRM) \citep{Hevner2004DSRM}. With DSRM, 7 guidelines can be listed specific to this reseach, inlcuding (1) Desiging the artifact of this project, which is the explainable medical AI framework. (2) The objective of this framework is to better diagnosis system in terms of performance an intepretability. (3) The result of this framework will be evaluated in 2 phases. The predictive model will firstly be evaluated by AUC. And the explanation for the prediction will be evaluated by a professional human radiologist in Portugal (4) The novelty of this work is the exploration of using different modalities during diagnosis, which may result a more robust and explanable framework. (5) The framework itself must be rigorously defined, formally represented, coherent, and internally consistent whether in devloping or evaluating phase. (6) the framework should be designed in an iterative and effective manner. (7) The result of this framwork should be presented to both technology-oriented and management-oriented audiences effectiviely. Moreove, according to the DSRM model \citep{Peffers2007DSRMForIS} proposed, our process model is shown as Figure \ref{fig: DSRMProcessModel}.


\begin{figure}[!h]
    \centering
    \includegraphics[width=.7\textwidth]{./images/DSRMProcessModel.png}
    \caption{DSRM Process model for this research.}
    \label{fig: DSRMProcessModel}
\end{figure}

\subsubsection{Research Plan}
% say that we will do evaluation with the help of radiologist. (Interview with radiologists.) (Assess our model output.). (Evaluate our framework)
%% Make Chart

% Say !!! We already have an established research collaboration with an experience radiologist from the main hospital in Portugal (This will be the subject that evaluate the work in all stages.) !!!

\subsection{Resources and Funding Required}


\subsection{Individual Contributions to the Research Team}

